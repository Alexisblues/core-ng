\chapter{Introduction}
\todo{more text}


\section{Key Features}

\section{Historic Background}
\todo{write some history}


\section{Concept presentation}
An overview of the DAPNET 2.0 concept is given in Fig. \ref{fig:concept_overview}.

\begin{figure}[htbp] 
  \centering
     \includegraphics[width=0.8\textwidth]{{"assets/Concept Version 2.0 Overview"}.png}
  \caption{Overview of DAPNET Clutering and Network Structure}
  \label{fig:concept_overview}
\end{figure}

The details of a single node implementation are shown in Fig. \ref{fig:node_details}.
\begin{figure}[htbp] 
  \centering
     \includegraphics[width=0.8\textwidth]{{"assets/Concept Version 2.0 Node Details"}.png}
  \caption{Node Details}
  \label{fig:node_details}
\end{figure}


\section{Transmitter Software}

\subsection{Unipager}

\subsection{DAPNET-Proxy}

\section{DAPNET Network}

\subsection{Overview and Concept}

\subsection{Used third-party Software}

\subsection{HAMCLOUD Description}
The HAMCLOUD is a virtual server combination of server central services on the HAMNET and provide short hop connectivity to deployed service on HAMNET towards the Internet. There are three data centers at Essen, Nürnberg and Aachen, which have high bandwidth interlinks over the DFN. There are address spaces for uni- and anycast services. How this concept is deployed is still tbd.
\todo{Define if uni- or anycast entry points will exist}
More information is here \url{https://www.swiss-artg.ch/fileadmin/Dokumente/HAMNET/HamCloud_-_Angebotene_Dienste_in_der_HamCloud.pdf} and here \url{http://hamnetdb.net/?m=as&q=hamcloud}.
\subsection{Rubric Handling Concept}

\subsection{Queuing Priority Concept}
