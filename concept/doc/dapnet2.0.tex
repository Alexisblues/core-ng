\documentclass[a4paper]{article}

%% Language and font encodings
\usepackage[english]{babel}
\usepackage[utf8x]{inputenc}
\usepackage[T1]{fontenc}

%% Sets page size and margins
\usepackage[a4paper,top=3cm,bottom=2cm,left=3cm,right=3cm,marginparwidth=1.75cm]{geometry}

%% Useful packages
\usepackage{amsmath}
\usepackage{graphicx}
\usepackage[colorinlistoftodos]{todonotes}
\usepackage[colorlinks=true, allcolors=blue]{hyperref}
\usepackage{listings,xcolor}

\setlength{\parindent}{0pt}
\setlength{\parskip}{1ex plus 0.5ex minus 0.2ex}

\lstset{
  basicstyle=\footnotesize\ttfamily,
  string=[s]{"}{"},
  stringstyle=\color{blue},
  comment=[l]{:},
  commentstyle=\color{black},
}

\title{DAPNET 2.0 Concept and Interface Definition}
\author{
Ralf Wilke, DH3WR\\
Thomas Gatzweiler, DL2IC\\
Phillip Thiel, DL6PT
}

\begin{document}
\maketitle
\begin{abstract}
This is the concept and interface description of the version 2 of the DAPNET. It's purpose in comparison to the first version released is a more robust clustering and network interaction solution to cope with the special requirements of IP connections over HAMNET which means that all network connections have to be considered with a WAN character resulting in unreliable network connectivity. In terms of consistence of the database, "eventually consistence" is considered to be the most reachable. There are "always right" database nodes inside the so called HAMCLOUD. In case of database conflicts, the version inside the HAMCLOUD cluster is always to be considered right.
\end{abstract}

\section{Introduction}
\todo{write some history}
\section{Concept presentation}

\subsection{Core functionality}

\subsection{HAMCLOUD presentation}
The HAMCLOUD is a virtual server combination to server central services on the HAMNET and provide short hop connectivity to deployed service on HAMNET towards the Internet. There are three data centers at Essen, Nürnberg and Aachen, which have high bandwidth interlinks over the DFN. There are address spaces for uni- and anycast services. How this concept is deployed is still tbd.
\todo{Define if uni- or anycast entry points will exist}
More information is here \url{https://www.swiss-artg.ch/fileadmin/Dokumente/HAMNET/HamCloud_-_Angebotene_Dienste_in_der_HamCloud.pdf} and here \url{http://hamnetdb.net/?m=as&q=hamcloud}.

\section{Interface definition}

\subsection{Transmitter sign-in, configuration and keep-alive}
If a transmitter wants to connect to DAPNET, the first step is to sign-in and show it's presence via a REST interface. Tis interface is also used for transmitter configuration like enabled timeslots and keep-alive polling.

\subsubsection{Authentication of all HTTP-Requests in this context}
All HTTP-requests issued from a transmitter have to send a valid HTTP authentication, which is checked against the CouchDB. It consists of the transmitter name and its AuthKey.


\subsubsection{Initial contact}

\texttt{POST /transmitter/bootstrap}
\begin{lstlisting}
{
  "callsign": "db0avr",
  "auth_key": "<secret>",
  "software": {
    "name": "UniPager",
    "version": "1.0.2"
  }
}
\end{lstlisting}

\subsubsection{Answers to the bootstrap REST call}

\texttt{200 OK}
\begin{lstlisting}
{
  "timeslots": [true, true, false, true, ...],
  "nodes": [
    {
      "host": "node1.ampr.org",
      "port": 4000,
      "reachable": true,
      "last_seen": "2018-07-03T07:43:52.783611Z",
      "response_time": 42
    }
  ]
}
\end{lstlisting}

\texttt{423 Locked}
\begin{lstlisting}
{
  "error": "Transmitter temporarily disabled by config."
}
\end{lstlisting}

\texttt{423 Locked}
\begin{lstlisting}
{
  "error": "Transmitter software type not allowed due to serious bug."
}
\end{lstlisting}


\subsubsection{Transmitter Heartbeat}
\texttt{POST /transmitter/heartbeat}
\begin{lstlisting}
{
  "callsign": "db0avr",
  "auth_key": "<secret>",
  "ntp_synced": true
}
\end{lstlisting}

\subsubsection{Answers to the heartbeat REST call}
\texttt{200 OK}
\begin{lstlisting}
{
  "status": "ok"
}
\end{lstlisting}

If network wants to assign new timeslots without disconnecting (for dynamic timeslots)

\texttt{200 OK}
\begin{lstlisting}
{
  "status": "ok",
  "timeslots": [true, true, false, ...],
  "valid_from": "2018-07-03T08:00:52.786458Z"
}
\end{lstlisting}

If network wants to initiate handover to other node

\texttt{503 Service unavailable}
\begin{lstlisting}
{
  "error": "Node not available, switch to other node."
}
\end{lstlisting}

\section{Visualization and management Website Connections}
There will a responsive website will be the main user interface. Dynamic data to be displayed is transfered by two methods:

\subsection{Polling based initial data dump}
When the website is called, the DAPNET REST API is use to gather the required data according to content just displayed. The API is defined in section \todo{write API docu}.

\subsection{Websocket based updates}
Once the actual status is known via the REST call, dynamic updates and telemetry data is send to the user's browser via websocket. Authentication is done by a authentication message by the browser right after connection. This procedure is the same as used in unipager. \todo{DL2IC: Könntest du das ergänzen?}

\subsubsection{Check authentication status}

\subsubsection{List of update messages}
The updates can be send combined in one message of each message separately. In any case, the \textit{transmittername} has to be sent to identify about what transmitter the data is.

\texttt{}
\begin{lstlisting}
{
"transmitterupdate" : {
	"name" : "db0abc",
	"status" : {
		"OnAir" : true,
		"ConfiguredIP": "123.4.3.2",
		"SoftwareType" : "Unipager",
		"SoftwareVersion" : "v1.2.3",
		"SoftwareType" : "MMDVM",
		"SoftwareVersion" : "20180504",
		"SoftwareType" : "DAPNET-Proxy",
		"SoftwareVersion" : "v2.3.4",
		"CPUHardwareType" : "Raspberry Pi 3B+"
		"RFHardware" : {
			"C9000" : {
				"Name" : "C9000",
				"UnipagerPowered" : true,
				"RPC-CardPowered" : false,
				"RPC-Version" : "XOS/2.23pre",
				"ArduinoPADummy" : true,
				"ArduinoPADummySettinginWatts : 123,
                "ArduinoPADummyPort" : "/devttyUSB0"
            },
			"Raspager" : {
				"Name" : "Raspager",
				"RaspagerMod" : 13,	
				"RaspagerPower" : 63,
				"ExternalPowerAmplifier": false,
				"RaspagerRFVersion" : "V2"
            }
			"Audio" : {
				"Name" : "Audio",
				"TXModel" : ["GM1200", "T7F", "GM340", "FREITEXT"],
				"AudioLevelUnipager" : 83,
				"TxDelayinMilliseconds" : 3,
			"RFM69" : {
				"Name" : "RFM69"
			},
			"MMDVM" : {
				"Name" : "MMDVM DUALHS...",
				"Mode" : "DAPNETExclusive",
				"Mode" : "DAPNETandDigitalVoice",
			}
		}
				
		"AX25" : {
			"ConnectionStatus" : "connected",
			"ConnectionStatus" : "connecting",
			"ConnectionStatus" : "disconnected"
		},
		"timeslots" : [true, false,...,	false]		
	},

	"RabbitMQ" : {
		"Connected" : true,
		"ConnectedtoNodeName" : "db0xyz"
		"ConnectedtoNodeIP" : "1.2.3.4"
		"ConnectedtoNodePort" : "1234"
		"ConnectedSince" : "<timestamp-format>"
		
		"NTPSynced" : true
		"NTPOffestMilliseconds" : 124
		"NTPServerUsedIP" : ["134.130.4.1", "12.2.3.2"]
	},
	"Telemetry" : {
		"QueuedMessages" : {
			"Total" : 1234,
			"Prio1": 1234,
			"Prio2": 1234,
			"Prio3": 1234,
			"Prio4": 1234,
			"Prio5": 1234,
			"Prio10": 1234
		},
		"PrefinedTemperaturesInCelsius" : {
			"AirInlet" : 12.2,
			"AirOutlet" : 14.2,
			"Transmitter" : 42.2,
			"PowerAmplifier" : 45.2,
			"CPU" : 93.2
		},

		"CustomTemperaturesInCelsius" : [
			{"Value" : 12.2, "Description" : "Aircon Inlet"},
			{"Value" : 16.2, "Description" : "Aircon Outlet"},
            {"Value" : 12.3, "Description" : "Fridge Next to Programmer"}
		],
		"PowerSupply" : {
			"OnBattery": false,
			"OnEmergencyPower: false
			"DCInputVoltageInVolt" : 12.4,
			"DCInputCurrentInAmpere" : 3.23
		},
		"RFoutput" : {
			"OutputPowerForwardinWatts : 12.2,
			"OutputPowerReturninWatts : 12.2,	
			"OutputVSWR : 1.2,
		},
}
\end{lstlisting}





%\subsection{How to include Figures}
%
%First you have to upload the image file from your computer using the upload link in the project menu. Then use the includegraphics command to include it in your document. Use the figure environment and the caption command to add a number and a caption to your figure. See the code for Figure \ref{fig:frog} in this section for an example.
%
%\subsection{How to add Tables}
%
%Use the table and tabular commands for basic tables --- see Table~\ref{tab:widgets}, for example.
%
%\begin{table}
%\centering
%\begin{tabular}{l|r}
%Item & Quantity \\\hline
%Widgets & 42 \\
%Gadgets & 13
%\end{tabular}
%\caption{\label{tab:widgets}An example table.}
%\end{table}
%
%\subsection{How to write Mathematics}
%
%\LaTeX{} is great at typesetting mathematics. Let $X_1, X_2, \ldots, X_n$ be a sequence of independent and identically distributed random variables with $\text{E}[X_i] = \mu$ and $\text{Var}[X_i] = \sigma^2 < \infty$, and let
%\[S_n = \frac{X_1 + X_2 + \cdots + X_n}{n}
%      = \frac{1}{n}\sum_{i}^{n} X_i\]
%denote their mean. Then as $n$ approaches infinity, the random variables $\sqrt{n}(S_n - \mu)$ converge in distribution to a normal $\mathcal{N}(0, \sigma^2)$.
%
%
%\subsection{How to create Sections and Subsections}
%
%Use section and subsections to organize your document. Simply use the section and subsection buttons in the toolbar to create them, and we'll handle all the formatting and numbering automatically.
%
%\subsection{How to add Lists}
%
%You can make lists with automatic numbering \dots
%
%\begin{enumerate}
%\item Like this,
%\item and like this.
%\end{enumerate}
%\dots or bullet points \dots
%\begin{itemize}
%\item Like this,
%\item and like this.
%\end{itemize}
%
%\subsection{How to add Citations and a References List}
%
%You can upload a \verb|.bib| file containing your BibTeX entries, created with JabRef; or import your \href{https://www.overleaf.com/blog/184}{Mendeley}, CiteULike or Zotero library as a \verb|.bib| file. You can then cite entries from it, like this: \cite{greenwade93}. Just remember to specify a bibliography style, as well as the filename of the \verb|.bib|.
%
%You can find a \href{https://www.overleaf.com/help/97-how-to-include-a-bibliography-using-bibtex}{video tutorial here} to learn more about BibTeX.
%
%We hope you find Overleaf useful, and please let us know if you have any feedback using the help menu above --- or use the contact form at \url{https://www.overleaf.com/contact}!
%
%\bibliographystyle{alpha}
%\bibliography{sample}

\end{document}
