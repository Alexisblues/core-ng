\chapter{Internal Programming Workflows}

\section{Sent calls}

\section{Add, edit, delete User}
\textbf{Show current users}
\begin{enumerate}
	\item Get current status via REST GET to /users on Core URL
	\item Handle updates via Websocket
\end{enumerate}

\textbf{Add and Edit User}
\begin{enumerate}
	\item If edit: Get current status via REST GET to /users/<username> on Core URL
	\item Show edit form and place data
	\item On save button event, send REST POST to /users/<username> on Core URL
\end{enumerate}
The core will update the CouchDB and generate a RabbitMQ administration message to inform all other nodes. This information is transmitted by the Stats and Websocket Micro-Service to all connected websocket clients to get them updated. This will also happen for the website instance emitting the edit request, so its content is also updated.


\textbf{Delete User}
\begin{enumerate}
	\item Ask "Are you sure?"
	\item If yes, sent REST DELETE to /users/<username> on Core URL
\end{enumerate}
The core will update the CouchDB and generate a RabbitMQ administration message to inform all other nodes. This information is transmitted by the Stats and Websocket Micro-Service to all connected websocket clients to get them updated. This will also happen for the website instance emitting the edit request, so its content is also updated.

\section{Add, edit, delete Subscriber}

\section{Add, edit, delete Node (tbd)}

\section{Add, edit, delete Transmitter}

\section{Implementation of Transmitter Groups}

\section{Add, edit, delete Rubrics}

\textbf{Show current configuration}
\begin{enumerate}
	\item Get current status via REST GET to /rubrics on Core URL
	\item Handle updates via websocket
\end{enumerate}

\textbf{Add and Edit rubrics}
\begin{enumerate}
	\item If edit: Get current status via REST GET to /rubrics/<rubricname> on Core URL
	\item Show edit form and place data
	\item On save button event, send REST POST to /users/<rubricname> on Core URL
\end{enumerate}
The core will update the CouchDB and generate a RabbitMQ administration message to inform all other nodes. This information is transmitted by the Stats and Websocket Micro-Service to all connected websocket clients to get them updated. This will also happen for the website instance emitting the edit request, so its content is also updated.


\textbf{Delete rubric}
\begin{enumerate}
	\item Ask "Are you sure?"
	\item If yes, sent REST DELETE to /users/<rubricname> on Core URL
\end{enumerate}
The core will update the CouchDB and generate a RabbitMQ administration message to inform all other nodes. This information is transmitted by the Stats and Websocket Micro-Service to all connected websocket clients to get them updated. This will also happen for the website instance emitting the edit request, so its content is also updated.

\section{Add, edit, delete Rubrics content}

\section{Add, edit, delete, assign Rubrics to Transmitter/-Groups}

\section{Ports and Loadbalacing Concept}

\section{Periodic Tasks (Scheduler)}

\section{Plugin Interface}

\section{Transmitter Connection}
Transmitter connections consist of two connections to a Node. A REST connection for initial announcement of a new transmitter, heartbeat messages and transmitter configuration and a RabbitMQ connection to receive the data to be transmitted.

The workflow for a transmitter connection is the following:
\begin{enumerate}
\item Announce new connecting transmitter via Core REST Interface (\ref{protocoldef:TXsignon}).
\item Get as response the transmitter configuration or an error message (\ref{protocoldef:TXconfig_or_error}).
\item Initiate RabbitMQ connection to get the data to be transmitted (\ref{protocoldef:RabbitMQ:Transmitters}).
\end{enumerate}

The authentication of the transmitter's REST calls consist of the transmitter name and its AuthKey, which is checked against the value in the CouchDB for this transmitter.




\section{Transmitter connections}
If a transmitter wants to connect to DAPNET, the first step is to sign-in and show it's presence via the Core REST interface. This interface is also used for transmitter configuration like enabled timeslots and keep-alive polling.

\subsection{Authentication of all HTTP-Requests in this context}
All HTTP-requests issued from a transmitter have to send a valid HTTP authentication, which is checked against the CouchDB. It consists of the transmitter name and its AuthKey.

\section{DAPNET-Proxy}