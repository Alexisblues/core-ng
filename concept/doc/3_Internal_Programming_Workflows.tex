\chapter{Internal Programming Workflows}

\section{Sent calls}

\section{Add, edit, delete User}
\subsection*{Show current users}
\begin{enumerate}
  \item Get current status via \texttt{GET /users} on Core URL
  \item Handle updates via Websocket
\end{enumerate}

\subsection*{Add and Edit User}
\begin{enumerate}
  \item If edit: Get current status via \texttt{GET /users/<username>} on Core URL
  \item Show edit form and place data
  \item On save button event, send \texttt{POST /users/<username>} on Core URL
\end{enumerate}

The core will update the CouchDB and generate a RabbitMQ administration message
to inform all other nodes. This information is transmitted by the Stats and
Websocket Micro-Service to all connected websocket clients to get them updated.
This will also happen for the website instance emitting the edit request, so its
content is also updated.


\subsection*{Delete User}
\begin{enumerate}
\item Ask "Are you sure?"
\item If yes, send \texttt{DELETE /users/<username>} on Core URL
\end{enumerate}

The core will update the CouchDB and generate a RabbitMQ administration message
to inform all other nodes. This information is transmitted by the Stats and
Websocket Micro-Service to all connected websocket clients to get them updated.
This will also happen for the website instance emitting the edit request, so its
content is also updated.

\section{Add, edit, delete Subscriber}

\section{Add, edit, delete Node (tbd)}

\section{Add, edit, delete Transmitter}

\section{Implementation of Transmitter Groups}

\section{Add, edit, delete Rubrics}

\textbf{Show current configuration}
\begin{enumerate}
  \item Get current status via \texttt{GET /rubrics} on Core URL
  \item Handle updates via websocket
\end{enumerate}

\textbf{Add and Edit rubrics}
\begin{enumerate}
\item If edit: Get current status via \texttt{GET /rubrics/<rubricname>} on Core URL
\item Show edit form and place data
\item On save button event, send \texttt{POST /users/<rubricname>} on Core URL
\end{enumerate}

The core will update the CouchDB and generate a RabbitMQ administration message
to inform all other nodes. This information is transmitted by the Stats and
Websocket Micro-Service to all connected websocket clients to get them updated.
This will also happen for the website instance emitting the edit request, so its
content is also updated.

\textbf{Delete rubric}
\begin{enumerate}
\item Ask "Are you sure?"
\item If yes, send \texttt{DELETE /users/<rubricname>} on Core URL
\end{enumerate}

The core will update the CouchDB and generate a RabbitMQ administration message
to inform all other nodes. This information is transmitted by the Stats and
Websocket Micro-Service to all connected websocket clients to get them updated.
This will also happen for the website instance emitting the edit request, so its
content is also updated.

\section{Add, edit, delete Rubrics content}

\section{Add, edit, delete, assign Rubrics to Transmitter/-Groups}

\section{Docker integration}
\label{internalprog:docker}
\todo{DL2IC: Docker Integration beschreiben}

\section{Microservices}
\label{internalprog:microservices}

A DAPNET node consists of serveral isolated microservices with different
responsibilities. Each microservice runs in container and is automatically
restarted if it should crash. Some microservices can be started in multiple
instances to fully utilize multiple cores. The access to the microservices is
proxied by a NGINX webserver which can also provide load balancing and caching.

\begin{center}
  \begin{tabular}{|l|l|} \hline
    REST endpoint & Microservice \\ \hline \hline

    \verb|*      /users/*| & Database Service \\
    \verb|*      /nodes/*| &  \\
    \verb|*      /rubrics/*| &  \\
    \verb|*      /subscribers/*| & \\
    \verb|*      /subscriber_groups/*| & \\
    \verb|DELETE /transmitters/:id| & \\
    \verb|PUT    /transmitters| & \\ \hline

    \verb|*      /calls/*| & Call Service \\ \hline

    \verb|*      /rubrics/:id/*| & Rubric Service \\ \hline

    \verb|GET    /transmitters| & Transmitter Service \\
    \verb|GET    /transmitters/:id| & \\
    \verb|POST   /transmitters/bootstrap| & \\
    \verb|POST   /transmitters/heartbeat| & \\ \hline

    \verb|POST   /cluster/discovery| & Cluster Service \\ \hline

    \verb|GET    /telemetry/*| & Telemetry Service \\ \hline
    \verb|WS     /telemetry| & \\ \hline

    \verb|WS     /changes| & Database Changes Service \\ \hline

    \verb|GET    /status/*| & Status Service \\ \hline

    \verb|GET    /statistics| & Statistics Service \\ \hline

    \verb|GET    /rabbitmq/*| & RabbitMQ Auth Service \\ \hline
  \end{tabular}
\end{center}

\subsection{Database Service}
\begin{itemize}
\item Proxies calls to the CouchDB database
\item Controls access to different database actions
\item Removes private/admin only fields from documents
\end{itemize}

\subsection{Call Service}
\begin{itemize}
\item Generates and publishes calls to RabbitMQ
\item Receives all calls from RabbitMQ
\item Maintains a database of all calls
\end{itemize}

\subsection{Rubric Service}
\begin{itemize}
\item Publishes rubric content as calls to RabbitMQ
\item Periodically publishes rubric names as calls to RabbitMQ
\end{itemize}

\subsection{Transmitter Service}
\begin{itemize}
\item Maintains a list of all transmitters and their current status
\end{itemize}

\subsection{Cluster Service}
\begin{itemize}
\item Maintains a list of known nodes and their current status
\item Manages federation between RabbitMQ queues
\item Manages replication between CouchDB databases
\end{itemize}

\subsection{Telemetry Service}
\begin{itemize}
\item Maintains the telemetry state of all transmitters
\item Forwards telemetry updates via websocket
\end{itemize}

\subsection{Database Changes Service}
\begin{itemize}
\item Forwards database changes via websocket
\end{itemize}

\subsection{Status Service}
\begin{itemize}
\item Periodically checks all other services and connections
\end{itemize}

\subsection{RabbitMQ Auth Service}
\begin{itemize}
\item Provides authentication for RabbitMQ against the CouchDB users database
\end{itemize}

\subsection{Time and Identification Service}
\begin{itemize}
\item Sends periodic time and identification messages to RabbitMQ
\end{itemize}

\section{Ports and Loadbalacing Concept}

\section{Periodic Tasks (Scheduler)}

\section{Plugin Interface}

\section{Transmitter Connection}
Transmitter connections consist of two connections to a Node. A REST connection
for initial announcement of a new transmitter, heartbeat messages and
transmitter configuration and a RabbitMQ connection to receive the data to be
transmitted.

The workflow for a transmitter connection is the following:
\begin{enumerate}
\item Announce new connecting transmitter via Core REST Interface (\ref{protcoldef:transmitters:bootstrap}).
\item Get as response the transmitter configuration or an error message (\ref{protcoldef:transmitters:bootstrap}).
\item Initiate RabbitMQ connection to get the data to be transmitted (\ref{protocoldef:RabbitMQ:Transmitters}).
\end{enumerate}

The authentication of the transmitter's REST calls consist of the transmitter
name and its AuthKey, which is checked against the value in the CouchDB for this
transmitter.

\section{Transmitter connections}
If a transmitter wants to connect to DAPNET, the first step is to sign-in and
show its presence via the Core REST interface. This interface is also used for
transmitter configuration like enabled timeslots and keep-alive polling.

\subsection{Authentication of all HTTP-Requests in this context}
All HTTP-requests issued from a transmitter have to send a valid HTTP
authentication, which is checked against the CouchDB. It consists of the
transmitter name and its AuthKey.

\todo{Da es sich bei den Anfragen um POST-Requests mit JSON Body handelt, wäre
es einfacher da den AuthKey mit dazu zu packen, so wie es auch schon in der
Protokoll-Definition umgesetzt ist.}

\section{DAPNET-Proxy}