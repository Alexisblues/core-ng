\chapter{Protocol Definitions}

\section{Core REST API}

\subsection{Transmitter sign-on, configuration and heartbeat}
\label{protocoldef:TXsignon}

\texttt{POST /transmitter/bootstrap}
\begin{lstlisting}
{
  "callsign": "db0avr",
  "auth_key": "<secret>",
  "software": {
    "name": "UniPager",
    "version": "1.0.2"
  }
}
\end{lstlisting}

\subsection{Answers to the bootstrap REST call}
\label{protocoldef:TXconfig_or_error}
\texttt{200 OK}
\begin{lstlisting}
{
  "timeslots": [true, true, false, true, ...],
  "nodes": [
    {
      "host": "node1.ampr.org",
      "port": 4000,
      "reachable": true,
      "last_seen": "2018-07-03T07:43:52.783611Z",
      "response_time": 42
    }
  ]
}
\end{lstlisting}

\texttt{423 Locked}
\begin{lstlisting}
{
  "error": "Transmitter temporarily disabled by config."
}
\end{lstlisting}

\texttt{423 Locked}
\begin{lstlisting}
{
  "error": "Transmitter software type not allowed due to serious bug."
}
\end{lstlisting}


\subsection{Transmitter Heartbeat}
\texttt{POST /transmitter/heartbeat}
\begin{lstlisting}
{
  "callsign": "db0avr",
  "auth_key": "<secret>",
  "ntp_synced": true
}
\end{lstlisting}

\subsubsection{Answers to the heartbeat REST call}
\texttt{200 OK}
\begin{lstlisting}
{
  "status": "ok"
}
\end{lstlisting}

If network wants to assign new timeslots without disconnecting (for dynamic timeslots)

\texttt{200 OK}
\begin{lstlisting}
{
  "status": "ok",
  "timeslots": [true, true, false, ...],
  "valid_from": "2018-07-03T08:00:52.786458Z"
}
\end{lstlisting}

If network wants to initiate handover to other node

\texttt{503 Service unavailable}
\begin{lstlisting}
{
  "error": "Node not available, switch to other node."
}
\end{lstlisting}

\section{RabbitMQ}
\label{protocoldef:RabbitMQ}
There are 4 exchanges on each RabbitMQ instance available:
\begin{enumerate}
\item dapnet.calls: Messages coming from other nodes but the node where the instance is running
\item dapnet.local\_calls: Messages coming from the local node instance
\item dapnet.telemetry: Messages containing telemetry from transmitters
\item dapnet.administration: Update on the CouchDB from other Nodes to inform the Websocket microservice about changes
\end{enumerate}


\subsection{Transmitters}
\label{protocoldef:RabbitMQ:Transmitters}

Valid Messages are:

\subsubsection{dapnet.calls}
\label{protocoldef:RabbitMQ:dapnet.calls}



\subsubsection{dapnet.calls}
\label{protocoldef:RabbitMQ:dapnet.local_calls}




\subsection{Telemetry}

\subsection{Administration}
\label{protocoldef:RabbitMQ:dapnet.administration}
If there is any update on the CouchDB, the other CouchDB instances are notified and updated automatically, but there is no trigger event to notify the websocket service about the change. As changes should be display immediately on the website or app, they have to be announced via websocket. In order to generate a trigger for changes in the CouchDB, the RabbitMQ exchange \textbf{dapnet.administration} is used and filled. All websocket microservices consume this exchange.


\subsubsection{Transmitter related}
\texttt{New transmitter added}
\begin{lstlisting}
{
  "type": "transmitter",
  "action" : "added",
  "name": "db0abc",
  "data" : {
  (Complete Data dump as stored in CouchDB)
  }
}
\end{lstlisting}

\texttt{Existing transmitter changed}
\begin{lstlisting}
{
  "type": "transmitter",
  "action" : "changed",
  "name": "db0abc",
  "data" : {
  (Complete Data dump as stored in CouchDB)
  }
}
\end{lstlisting}

\texttt{Transmitter deleted}
\begin{lstlisting}
{
  "type": "transmitter",
  "action" : "deleted",
  "name": "db0abc"
}
\end{lstlisting}

\subsubsection{User related}
\texttt{New User added}
\begin{lstlisting}
{
  "type": "user",
  "action" : "added",
  "name": "db1abc",
  "data" : {
  (Complete Data dump as stored in CouchDB)
  }
}
\end{lstlisting}

\texttt{Existing user changed}
\begin{lstlisting}
{
  "type": "user",
  "action" : "changed",
  "name": "db1abc",
  "data" : {
  (Complete Data dump as stored in CouchDB)
  }
}
\end{lstlisting}

\texttt{User deleted}
\begin{lstlisting}
{
  "type": "user",
  "action" : "deleted",
  "name": "db1abc"
}
\end{lstlisting}


\subsubsection{Rubric related}
\texttt{New Rubric added}
\begin{lstlisting}
{
  "type": "rubric",
  "action" : "added",
  "id": "...",
  "data" : {
  (Complete Data dump as stored in CouchDB)
  }
}
\end{lstlisting}

\texttt{Existing rubric changed}
\begin{lstlisting}
{
  "type": "user",
  "action" : "changed",
  "id": "...",
  "data" : {
  (Complete Data dump as stored in CouchDB)
  }
}
\end{lstlisting}

\texttt{Rubric deleted}
\begin{lstlisting}
{
  "type": "user",
  "action" : "deleted",
  "id": "..."
}
\end{lstlisting}

\subsubsection{Rubric content related}
\todo{Check against CouchDB structure}

\texttt{New Rubric content added}
\begin{lstlisting}
{
  "type": "rubric_content",
  "action" : "added",
  "id": "...??",
  "data" : {
  (Complete Data dump of all ten rubric messages as stored in CouchDB)
  }
}
\end{lstlisting}

\texttt{Existing rubric changed}
\begin{lstlisting}
{
  "type": "rubric_content",
  "action" : "changed",
  "id": "...",
  "data" : {
  (Complete Data dump of all ten rubric messages as stored in CouchDB)
  }
}
\end{lstlisting}

\texttt{Rubric content deleted}
\begin{lstlisting}
{
  "type": "rubric_content",
  "action" : "deleted",
  "id": "..."
  "data" : {
  (Complete Data dump of all ten rubric messages as stored in CouchDB, some may be empty)
  }
}
\end{lstlisting}


\subsection{MQTT API for third-party consumers}


\section{Telemetry}
\label{protocoldef:telemetry}
Telemetry is sent from transmitters to the RabbitMQ exchange \textbf{dapnet.telemetry} as defined in section \ref{protocoldef:RabbitMQ}. It is also used in the same way on the websocket API to inform the website and the app about the telemetry in real-time in section \ref{protocoldef:websocketapi:telemetry}.

This is sent every minute in complete. If there are changes, just a subset is sent. The \"name\" key is always mandatory.
\begin{lstlisting}
{
    "name": "db0acb",
    "OnAir" : true,
    "Telemetry" : {
        "ConnectionStatus": {
            "Connected" : true,
            "ConnectedtoNodeName" : "db0xyz"
            "ConnectedtoNodeIP" : "1.2.3.4"
            "ConnectedtoNodePort" : 1234
            "ConnectedSince" : "<timestamp-format>",
	    	"NTPSynced" : true,
            "NTPOffestMilliseconds" : 124
            "NTPServerUsedIP" : ["134.130.4.1", "12.2.3.2"]
        },
		"QueueStatus" : {
        	"QueuedMessages" : {
                "Total" : 1234,
                "Prio1": 1234,
                "Prio2": 1234,
                "Prio3": 1234,
                "Prio4": 1234,
                "Prio5": 1234,
                "Prio10": 1234
	        }
		},
        "PrefinedTemperatures" : {
            "Unit" : "C" | "F" | "K",
         
            "AirInlet" : 12.2,
            "AirOutlet" : 14.2,
            "Transmitter" : 42.2,
            "PowerAmplifier" : 45.2,
            "CPU" : 93.2,
            "PowerSupply" : 32.4
        },
        "CustomTemperatures" : {
            "Unit" : "C" | "F" | "K",
            [
                {"Value" : 12.2, "Description" : "Aircon Inlet"},
                {"Value" : 16.2, "Description" : "Aircon Outlet"},
                {"Value" : 12.3, "Description" : "Fridge Next to Programmer"}
            ],
        },
        "PowerSupply" : {
            "OnBattery": false,
            "OnEmergencyPower": false,
            "DCInputVoltage" : 12.4,
            "DCInputCurrent" : 3.23
        },
        "RFoutput" : {
            "OutputPowerForwardinWatts": 12.2,
            "OutputPowerReturninWatts" : 12.2,
            "OutputVSWR" : 1.2
        }
	"transmitterconfiguration" : {
		"ConfiguredIP": "123.4.3.2",
		"timeslots" : [true, false,...,	false],
		"SoftwareType" : "Unipager" | "MMDVM" | "DAPNET-Proxy",
        "SoftwareVersion" : "v1.2.3", | "20180504" | "v2.3.4",
		"CPUHardwareType" : "Raspberry Pi 3B+"
		"RFHardware" : {
			"C9000" : {
				"UnipagerPowered" : true,
				"ArduinoPADummy" : true,
				"ArduinoPADummySettinginWatts" : 123,
				"ArduinoPADummyPort" : "/devttyUSB0"
or
				"RPC-CardPowered" : false,
				"RPC-Version" : "XOS/2.23pre",
			},
            "Raspager" : {
				"RaspagerMod" : 13,
				"RaspagerPower" : 63,
				"ExternalPowerAmplifier": false,
				"RaspagerRFVersion" : "V2"
			},
			"Audio" : {
				"TXModel" : ["GM1200", "T7F", "GM340", "FREITEXT"],
				"AudioLevelUnipager" : 83,
				"TxDelayinMilliseconds" : 3
			},
			"RFM69" : {
				"Port" : "/dev/ttyUSB0"
			},
			"MMDVM DualHS..." : {
				"DAPNETExclusive" : true
			}
		},
		"DAPNET_Proxy" : {
			"ConnectionStatus" : "connected",
			"ConnectionStatus" : "connecting",
			"ConnectionStatus" : "disconnected"
		}
	}
}
\end{lstlisting}


\section{Statistic and Telemetry REST API}

\section{Websocket API}
\label{protocoldef:websocketapi}
The idea is to provide an API for the website and the app to display real-time information without the need of polling. A websocket server is listing to websocket connections. Authentication is done by a custom JOSN handshake. The connection might be encrypted with SSL if using the Internet or plain if using HAMNET.

\subsection{Telemetry}
\label{protcoldef:websocketapi:telemetry}
The data is the same as received from the \textbf{dapnet.telemetry} exchange from the RabbitMQ instance. It is defined in section \ref{protocoldef:telemetry}.

\subsection{Administration}
\label{protcoldef:websocketapi:administration}
The data is the same as received from the \textbf{dapnet.administation} exchange from the RabbitMQ instance. It i defined in section \ref{protocoldef:RabbitMQ:dapnet.administration}.

\section{CouchDB Documents and Structure}
\todo{als Tabelle darstellen}
\subsection{Users}

\begin{lstlisting}
{
	"_id": "dl1abc",
	"password": "some hash",
    "email": "user@example.com",
    "admin": true,
    "enabled":true,
    "created_on":<DATETIME>,
    "last_change_by":"dh3wr",
    "email_valid":true
    "avatar_picture": <couchdb attachment??>
}
\end{lstlisting}

\subsection{Nodes}
\begin{lstlisting}
{
	"_id" : "db0abc",
	"status" : "OFFLINE" | "ONLINE" | "ERROR",
	"last_update" : DATETIME,
    "version" : "1.2.3",
    "ip_address" : "1.2.3.4",
    "latitude" : 34.123456,
    "longitude" : -23.123456,
    "hamcloudnode" : true,
    "owners" : ["dl1abc","dh3wr","dl2ic"],
    "avatar_picture": <couchdb attachment??>
}
\end{lstlisting}

\subsection{Transmitters}

\begin{lstlisting}
{
	"_id" : "db0abc",
    "auth_keys" : "hdjaskhdlj",
    "enabled" : true,
	"status" : "UNKNOWN" | "OFFLINE" | "ONLINE",
	"site_type" : "PERSONAL" | "WIDERANGE",
	"aprs_reporting_enabled" : true,
    "last_update" : "<DATETIME>",
	"ast_connect" : "<DATETIME>",
	"connected_since" : "<DATETIME>",
    "ip_address" : "1.2.3.4",
    "device_type" : "Unipager",
    "device_version" : "1.3.2",
    "latitude" : 23.123456,
    "longitude" : -31.123456
    "rf_power_watt": 12.3,
    "cable_loss_db" : 4.2,
    "antenna_gain_dbi" : 2.34,
	"antenna_agl_m" : 23.4,
	"antenna_type" : "OMNI" | "DIRECTIONAL",
	"antenna_direction" : 123.2,
    "owners" : ["dl1abc","dh3wr","dl2ic"],
    "groups" : ["dl-hh", "dl-all"],
    "frequency_MHz" : 439.9875,
    "emergency_power_available" : false,
    "infinite_emergency_power" : false,
	"emergency_power_duration_hours" : 23.0        
    "antenna_pattern" : <couchDB attachment>,
    "avatar_picture" : <couchDB attachment>
}
\end{lstlisting}

 
\subsection{Subscribers}
\todo{check if [] is valid JSON}
\begin{lstlisting}
{
	"_id" : "dl1abc",
	"description" : "Peter",
	"pagers" : [
		{
    		"ric" : {123456, "A"} ... "ric" : {123456, "D"},

			"uuid" : "0023-1233-aefe-1234-3423-9812",
    		"name" : "Peters Alphapoc",

			"type" : "UNKNOWN" | "Skyper" | "AlphaPoc" | "QUIX" | "Swissphone" | "SCALL_XT" | "Birdy"

			"is_enabled" : true
     	},
     	...
     ],
    "owner" : ["dh3wr", "dl1abc"]
}
\end{lstlisting}

\subsection{Subscriber Groups}

\begin{lstlisting}
{
	"_id" : "ov-G01",
	"description" : "Ortverband Aachen",
	"member_subscribers" : ["dl1abc", "dh3wr"],
    "owner" : ["dh3wr", "dl1abc"],
}
\end{lstlisting}

\subsection{Rubrics List}
\label{rubric_list}

\begin{lstlisting}
{
   	"uuid" : "<UUID>"
    "number" : 14,
    "description" : "Wetter DL-HH",
    "label" : "WX DL-HH",
    "transmitter_groups" : ["dl-hh","dl-ns"],
    "transmitters" : ["db0abc"],
	"cyclic_transmit_enabled" : true,
    "cyclic_transmit_interval_minutes": 123,
 	"owner" : ["dh3wr", "dl1abc"]
}
\end{lstlisting}

\subsection{Rubric's content}
<UUID> of rubric (as defined in \ref{rubric_list})

\begin{lstlisting}
{
	"uuid" : "<UUID>",
	["content_message1",..,"content_message10"],
}
\end{lstlisting}